%%%%%%%%%%%%%%%%%%%%%%%%%%%%%%%%%%%%%%%%%
% Original author:
% Ted Pavlic (http://www.tedpavlic.com)
%%%%%%%%%%%%%%%%%%%%%%%%%%%%%%%%%%%%%%%%%

%----------------------------------------------------------------------------------------
%	PACKAGES AND OTHER DOCUMENT CONFIGURATIONS
%----------------------------------------------------------------------------------------

\documentclass{article}

\usepackage{fancyhdr} % Required for custom headers
\usepackage{lastpage} % Required to determine the last page for the footer
\usepackage{extramarks} % Required for headers and footers
\usepackage[usenames,dvipsnames]{color} % Required for custom colors
\usepackage{graphicx} % Required to insert images
\usepackage{listings} % Required for insertion of code
\usepackage{courier} % Required for the courier font
\usepackage[spanish]{babel}


% Margins
\topmargin=-0.45in
\evensidemargin=0in
\oddsidemargin=0in
\textwidth=6.5in
\textheight=9.0in
\headsep=0.25in

\linespread{1.1} % Line spacing

% Set up the header and footer
\pagestyle{fancy}
\lhead{\hmwkAuthorName} % Top left header
\chead{\hmwkClass\ (\hmwkClassInstructor\ \hmwkClassTime): \hmwkTitle} % Top center head
\rhead{\firstxmark} % Top right header
\lfoot{\lastxmark} % Bottom left footer
\cfoot{} % Bottom center footer
\rfoot{Page\ \thepage\ of\ \protect\pageref{LastPage}} % Bottom right footer
\renewcommand\headrulewidth{0.4pt} % Size of the header rule
\renewcommand\footrulewidth{0.4pt} % Size of the footer rule

\setlength\parindent{0pt} % Removes all indentation from paragraphs

%----------------------------------------------------------------------------------------
%	CODE INCLUSION CONFIGURATION
%----------------------------------------------------------------------------------------

\definecolor{MyDarkGreen}{rgb}{0.0,0.4,0.0} % This is the color used for comments
\lstloadlanguages{C, Java} % Load Ansi C syntax for listings, for a list of other languages supported see: 
% ftp://ftp.tex.ac.uk/tex-archive/macros/latex/contrib/listings/listings.pdf
\lstset{language=C,
        frame=single, % Single frame around code
        basicstyle=\small\ttfamily, % Use small true type font
        keywordstyle=[1]\color{Blue}\bf, % Perl functions bold and blue
        keywordstyle=[2]\color{Purple}, % Perl function arguments purple
        keywordstyle=[3]\color{Blue}\underbar, % Custom functions underlined and blue
        identifierstyle=, % Nothing special about identifiers                                         
        commentstyle=\usefont{T1}{pcr}{m}{sl}\color{MyDarkGreen}\small, % Comments small dark green courier font
        stringstyle=\color{Purple}, % Strings are purple
        showstringspaces=false, % Don't put marks in string spaces
        tabsize=5, % 5 spaces per tab
        %
        % Put standard Perl functions not included in the default language here
        morekeywords={rand},
        %
        % Put Perl function parameters here
        morekeywords=[2]{on, off, interp},
        %
        % Put user defined functions here
        morekeywords=[3]{test},
       	%
        morecomment=[l][\color{Blue}]{...}, % Line continuation (...) like blue comment
        numbers=left, % Line numbers on left
        firstnumber=1, % Line numbers start with line 1
        numberstyle=\tiny\color{Blue}, % Line numbers are blue and small
        stepnumber=5 % Line numbers go in steps of 5
}

% Creates a new command to include a Ansi C source code, the first parameter is the filename of the script (without .pl), the second parameter is the caption
\newcommand{\ansic}[2]{
\begin{itemize}
\item[]\lstinputlisting[caption=#2,label=#1]{#1.c}
\end{itemize}
}

%----------------------------------------------------------------------------------------
%	DOCUMENT STRUCTURE COMMANDS
%	Skip this unless you know what you're doing
%----------------------------------------------------------------------------------------

% Header and footer for when a page split occurs within a problem environment
\newcommand{\enterProblemHeader}[1]{
\nobreak\extramarks{#1}{#1 continued on next page\ldots}\nobreak
\nobreak\extramarks{#1 (continued)}{#1 continued on next page\ldots}\nobreak
}

% Header and footer for when a page split occurs between problem environments
\newcommand{\exitProblemHeader}[1]{
\nobreak\extramarks{#1 (continued)}{#1 continued on next page\ldots}\nobreak
\nobreak\extramarks{#1}{}\nobreak
}

\setcounter{secnumdepth}{0} % Removes default section numbers
\newcounter{homeworkProblemCounter} % Creates a counter to keep track of the number of problems

\newcommand{\homeworkProblemName}{}
\newenvironment{homeworkProblem}[1][Assignment \arabic{homeworkProblemCounter}]{ % Makes a new environment called homeworkProblem which takes 1 argument (custom name) but the default is "Problem #"
\stepcounter{homeworkProblemCounter} % Increase counter for number of problems
\renewcommand{\homeworkProblemName}{#1} % Assign \homeworkProblemName the name of the problem
\section{\homeworkProblemName} % Make a section in the document with the custom problem count
\enterProblemHeader{\homeworkProblemName} % Header and footer within the environment
}{
\exitProblemHeader{\homeworkProblemName} % Header and footer after the environment
}

\newcommand{\problemAnswer}[1]{ % Defines the problem answer command with the content as the only argument
\noindent\framebox[\columnwidth][c]{\begin{minipage}{0.98\columnwidth}#1\end{minipage}} % Makes the box around the problem answer and puts the content inside
}

\newcommand{\homeworkSectionName}{}
\newenvironment{homeworkSection}[1]{ % New environment for sections within homework problems, takes 1 argument - the name of the section
\renewcommand{\homeworkSectionName}{#1} % Assign \homeworkSectionName to the name of the section from the environment argument
\subsection{\homeworkSectionName} % Make a subsection with the custom name of the subsection
\enterProblemHeader{\homeworkProblemName\ [\homeworkSectionName]} % Header and footer within the environment
}{
\enterProblemHeader{\homeworkProblemName} % Header and footer after the environment
}

%----------------------------------------------------------------------------------------
%	NAME AND CLASS SECTION
%----------------------------------------------------------------------------------------

\newcommand{\hmwkTitle}{Robot M\'ovil} % Assignment title
\newcommand{\hmwkDueDate}{Noviembre 2016} % Due date
\newcommand{\hmwkClass}{Programming Technologies} % Course/class
\newcommand{\hmwkClassTime}{Embedded Programming} % Class/lecture time
\newcommand{\hmwkAuthorName}{} % Teacher/lecturer
\newcommand{\hmwkClassInstructor}{Jos\'e A. Avi\~na - } % Your name

%----------------------------------------------------------------------------------------
%	TITLE PAGE
%----------------------------------------------------------------------------------------

\title{
\vspace{2in}
\textmd{\textbf{\hmwkClass:\ \hmwkTitle}}\\
\normalsize\vspace{0.1in}\small{Due\ on\ \hmwkDueDate}\\
\vspace{0.1in}\large{\textit{\hmwkClassInstructor\ \hmwkClassTime}}
\vspace{3in}
}

\author{\textbf{\hmwkAuthorName}}
\date{} % Insert date here if you want it to appear below your name

%----------------------------------------------------------------------------------------

\begin{document}

\maketitle

%----------------------------------------------------------------------------------------
%	TABLE OF CONTENTS
%----------------------------------------------------------------------------------------

%\setcounter{tocdepth}{1} % Uncomment this line if you don't want subsections listed in the ToC

\newpage
\tableofcontents
\newpage

%----------------------------------------------------------------------------------------
%	Assingment 1
%----------------------------------------------------------------------------------------


\begin{homeworkProblem}
 Construir un Robot M\'ovil basado en un Microcontrolador ATMEGAx y compuesto por:

  \begin{itemize}
    \item Chasis.
    \item Dos motoreductores de CD.
    \item Rueda loca.
    \item Sensor de luz (fotoceldas).  
    \item Cuatro bater\'ias de 1.5V para energizar la tarjeta arduino.
    \item Una bater\'ia para energizar los motoreductores.  
  \end{itemize}
La funcionalidad del Robot M\'ovil se programar\'a en lenguaje C. Lo cual implica el actuado o control de los motores y el control del sensor de luz.

\end{homeworkProblem}


A partir de la definici\'on de un mapa conformado por \'areas de edifcios, calles y avenidas:
\begin{homeworkProblem}
Implementar el algoritmo de b\'usqueda \emph{Primero el Mejor} para obtener la mejor ruta de tr\'ansito desde un punto A a un punto B; predefinidos en el Mapa. 


Observaci\'on: el heur\'istico a emplear para establecer la mejor opci\'on por visitar, es seleccionar aquella coordenada del mapa conocida que te acerque m\'as al punto final B. De esta manera, se deben definir puntos (coordenadas) en el mapa para que \'estas representen estados por visitar y sobre los cuales transitar hacia el punto B.  
\end{homeworkProblem}

\begin{homeworkProblem}
A partir del uso del Robot M\'ovil, basado en el microcontrolador ATMEGAX (Arduino), resolver la mejor ruta desde el punto
inicial A hacia el punto final B donde el Espacio de Estados se construye sobre una Matriz de 10 x 10 mosaicos:

\begin{itemize}  
 \item El punto inicial A se situar\'a siempre sobre la primera columna (cualquier fila) de la Matriz M.
 \item El punto final B se situar\'a siempre en la columna final (cualquier fila) de la Matriz M.
 \item Sobre la Matriz M se podr\'an trazar din\'amicamente diversas rutas que unan los puntos A y B.
 \item Utilizar superficies claras y obscuras para facilitar la lectura del color por parte del Robot M\'ovil. 
 \item La Matriz puede conformarse por mosaicos en el aula de clase, cubriendo de color negro aquellos mosaicos que 
conformar\'an las diversas rutas que unen los puntos A y B.   
\end{itemize}  
Finalmente, el telecontrol v\'ia bluetooth (BlueZ linux/App Inventor 2):

\begin{itemize}  
 \item Controlar\'a el  Robot M\'ovil para recorrer completamente la Matriz M:
  \begin{itemize}  
    \item Utilizando el sensor de luz, el Robot M\'ovil debe leer todos y cada uno de los mosaicos de la Matriz;
          diferenciando entre el color claro (0) y oscuro (1).
    \item Al concluir este primer recorrido, el Robot M\'ovil debe transmitir a linux los claros y oscuros sensados a fin
            de que el computador construya el Espacio de Estados (conexiones en la Matriz M) y asimismo calcule la mejor 
            ruta que una al punto A con el punto B.  
    \item Finalmente, mostrar la mejor ruta en la terminal linux.   
  \end{itemize}  
\end{itemize}  

\end{homeworkProblem}


\end{document}
